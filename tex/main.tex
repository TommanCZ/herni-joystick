\documentclass[hidelinks,12pt]{report}
\usepackage[czech]{babel}
\usepackage{setspace} % umožňuje nastavení řádkování
\usepackage{hyperref} % clickable odkazy
\usepackage{csquotes} % citování v odstavci
\usepackage{blindtext} % lipsum
\usepackage{amsmath} % matika
\usepackage{enumitem}
\usepackage{fontspec} % fonty: roman font je libre times new roman, sans-serif je libre arial
	\setmainfont[Scale=MatchLowercase,
	BoldFont = * Bold,
	ItalicFont = * Italic,
	BoldItalicFont = * Bolditalic]
	{Liberation Serif}
	\setsansfont[Scale=MatchLowercase,
	BoldFont = * Bold,
	ItalicFont = * Italic,
	BoldItalicFont = * Bolditalic]
	{Liberation Sans}
\usepackage{sectsty} % všechny nadpisy jsou sans serif
	\allsectionsfont{\sffamily}
\usepackage{graphicx}
	\graphicspath{ {obrazky/} } % path to images dir
\usepackage{natbib}
	\setcitestyle{authoryear,notesep={,~}}
	\bibliographystyle{literatura/apa.bst} % APA citace
%	\bibliographystyle{abbrvnat} % zapnoit aby soupis literatury používal zkratky

\def\nazevprace{  } % Název práce
\def\podnazev{  } % Rozšiřující název práce
\def\jmenoprijmeni{  } % Jméno Příjmení
\def\predmet{Informatika} % Předmět, ze kterého maturuješ
\def\skola{Gymnázium Jana Keplera}
\def\trida{4.A}
\def\mesto{Praha}
\def\rok{2025}

\usepackage[a4paper,width=150mm,top=25mm,bottom=25mm]{geometry}
\usepackage{fancyhdr}
	\setlength{\headheight}{14.49998pt}
	\pagestyle{fancy}
	\fancyhead{}
	\fancyhead[RO,LE]{\nazevprace}
	\fancyfoot{}
	\fancyfoot[LE,RO]{\thepage}
%	\fancyfoot[LO,CE]{Kapitola \thechapter}
%	\fancyfoot[CO,RE]{\jmenoprijmeni}

\begin{document}
\thispagestyle{plain}
\begin{titlepage}
    \begin{center}
            
        \normalsize
        \textsf{Gymnázium Jana Keplera}\\
        \vspace{1cm}
        \includegraphics[scale=0.4]{{kepler.png}}
        \vspace{3.5cm}
        
        
        \Huge
        \textbf{\textsf{\nazevprace}}
        
        \vspace{0.4cm}
        \Large
        \textsf{\podnazev}
        
        
        \vspace{2.5cm}
        \large
        \textit{\textsf{Maturitní práce}\\
        \textsf{k předmětu Humanitní studia}}
        \vspace{2cm}
            
        \vfill
            
        \normalsize
        \textbf{\textsf{\jmenoprijmeni}}\\
        \textsf{\trida}\\
        \textsf{\mesto}\\
        \textsf{\rok}
    \end{center}
\end{titlepage} }

\setstretch{1.5} % řádkování 1.5, až po titulní straně

\tableofcontents

\addcontentsline{toc}{chapter}{Úvod}
\chapter*{Úvod}
\input{kapitoly/uvod}

\chapter{Teoretická část}
\input{kapitoly/teoreticka-cast}

\chapter{Praktická část}
\input{kapitoly/prakticka-cast}

\newpage
\addcontentsline{toc}{chapter}{Literatura}
\bibliography{literatura/refs.bib}

\end{document}